\section{Управление дебиторской задолженностью}

Дебиторская задолженность выступает главной составляющей оборотного капитала компании. Ее можно разделить по следующим видам:
\begin{itemize}
	\item за товары и услуги (80--90 \% всего объема дебиторской задолженности);
	\item по полученным векселям;
	\item по расчетам с бюджетом;
	\item по расчетам с персоналом;
	\item прочие виды.
\end{itemize}

На уровень дебиторской задолженности оказывают влияние такие факторы как: вид выпускаемой продукции, емкость рынка, степень насыщенности рынка данной продукцией, принятая в организации система расчетов и др.

Задача финансового менеджмента --- эффективное управление дебиторской задолженностью с целью оптимизации ее размера и обеспечения своевременной индексации долга.

Управление дебиторской задолженностью включает следующие основные этапы:
\begin{enumerate}
	\item проведение анализа задолженности в предшествующем периоде;
	\item 
\end{enumerate}
