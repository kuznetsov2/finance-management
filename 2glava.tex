\section{Управление дебиторской задолженностью}

Дебиторская задолженность выступает главной составляющей оборотного капитала компании. Ее можно разделить по следующим видам:
\begin{itemize}
	\item за товары и услуги (80--90 \% всего объема дебиторской задолженности);
	\item по полученным векселям;
	\item по расчетам с бюджетом;
	\item по расчетам с персоналом;
	\item прочие виды.
\end{itemize}

На уровень дебиторской задолженности оказывают влияние такие факторы как: вид выпускаемой продукции, емкость рынка, степень насыщенности рынка данной продукцией, принятая в организации система расчетов и др.

Задача финансового менеджмента --- эффективное управление дебиторской задолженностью с целью оптимизации ее размера и обеспечения своевременной индексации долга.

Управление дебиторской задолженностью включает следующие основные этапы:
\begin{enumerate}
	\item проведение анализа задолженности в предшествующем периоде;
	\item формирование принципов кредитной политики по отношению к покупателям продукции;
	\item разработка процедуры индексации дебиторской задолженности и построение систем контроля движения и своевременного  погашения дебиторской задолженности;
	\item разработку путей сокращения дебиторской задолженности.
\end{enumerate}

Величину и формирование дебиторской задолженности на конкретный период определяют на основе соотношения:
\[ \text{ДЗ}_\text{нп} + \text{РП}_\text{отгр} =\text{РП}_\text{опл} + \text{ДЗ}_\text{кп} , \]
отсюда
\[ \text{ДЗ}_\text{кп} = \text{РП}_\text{отгр} \minus \text{РП}_\text{опл} + \text{ДЗ}_\text{нп} , \]
где $ \text{ДЗ}_\text{нп} ,\  \text{ДЗ}_\text{кп}$ — дебиторская задолженность на начало и конец периода;\\
$\text{РП}_\text{отгр}$  — величина отгруженной продукции;\\
$\text{РП}_\text{опл}$ — величина оплаченной продукции.


