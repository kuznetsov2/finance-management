\section{Управление дебиторской задолженностью}

Дебиторская задолженность выступает главной составляющей оборотного капитала компании. Ее можно разделить по следующим видам:
\begin{itemize}
	\item за товары и услуги (80--90 \% всего объема дебиторской задолженности);
	\item по полученным векселям;
	\item по расчетам с бюджетом;
	\item по расчетам с персоналом;
	\item прочие виды.
\end{itemize}

На уровень дебиторской задолженности оказывают влияние такие факторы как: вид выпускаемой продукции, емкость рынка, степень насыщенности рынка данной продукцией, принятая в организации система расчетов и др.

Задача финансового менеджмента --- эффективное управление дебиторской задолженностью с целью оптимизации ее размера и обеспечения своевременной индексации долга.

Управление дебиторской задолженностью включает следующие основные этапы:
\begin{enumerate}
	\item проведение анализа задолженности в предшествующем периоде;
	\item формирование принципов кредитной политики по отношению к покупателям продукции;
	\item разработка процедуры индексации дебиторской задолженности и построение систем контроля движения и своевременного  погашения дебиторской задолженности;
	\item разработку путей сокращения дебиторской задолженности.
\end{enumerate}

Величину и формирование дебиторской задолженности на конкретный период определяют на основе соотношения:
\[ \text{ДЗ}_\text{нп} + \text{РП}_\text{отгр} =\text{РП}_\text{опл} + \text{ДЗ}_\text{кп} , \]
отсюда
\[ \text{ДЗ}_\text{кп} = \text{РП}_\text{отгр} - \text{РП}_\text{опл} + \text{ДЗ}_\text{нп} , \]
где $ \text{ДЗ}_\text{нп} ,\  \text{ДЗ}_\text{кп}$ — дебиторская задолженность на начало и конец периода;\\
$\text{РП}_\text{отгр}$  — величина отгруженной продукции;\\
$\text{РП}_\text{опл}$ — величина оплаченной продукции.

В современной коммерческой и финансовой практике реализация продукции в кредит (с отсрочкой платежа за нее) получила широкое распространение как в нашей стране, так и в странах с развитой рыночной экономикой.
Формирование принципов кредитной политики отражает условия этой практики и направлено на повышение эффективности операционной и финансовой деятельности компании.

В процессе формирования принципов кредитной политики по отношению к покупателям продукции решаются два основных вопроса: в каких формах осуществлять реализацию продукции в кредит; какой тип кредитной политики следует избрать компании.

В зависимости от уровня доходности и вероятности риска кредитной деятельности различают три принципиальных ее типа: консервативный, умеренный и агрессивный.

Консервативный (или жесткий) тип кредитной политики организации направлен на минимизацию кредитного риска.
Осуществляя этот тип кредитной политики предприятие не стремится к получению высокой дополнительной прибыли за счет расширения объема реализации продукции.

Умеренный тип кредитной политики организации ориентируется на средний уровень кредитного риска при продаже продукции с отсрочкой платежа.

Агрессивный (или мягкий) тип кредитной политики --- это расширение объема реализации продукции в кредит, не считаясь с высоким уровнем кредитного риска.

В процессе выбора типа кредитной политики должны учитываться следующие основные факторы: общее состояние экономики, определяющее финансовые возможности покупателей;
уровень их платежеспособности; сложившуюся конъюнктуру товарного рынка, состояние спроса на продукцию компании; потенциальную способность компании наращивать объем производства продукции при расширении возможностей ее реализации за счет предоставления кредита; правовые условия обеспечения взыскания дебиторской задолженности; финансовые возможности компании в части отвлечения средств в текущую дебиторскую задолженность; финансовый менталитет менеджмента компании, его отношение к уровню допустимого риска в процессе осуществления хозяйственной деятельности.

Определяя тип кредитной политики, следует иметь в виду, что жесткий (консервативный) ее вариант отрицательно влияет на рост объема операционной деятельности компании и формирование устойчивых коммерческих связей, в то время как мягкий (агрессивный) ее вариант может вызвать чрезмерное отвлечение финансовых средств, снизить уровень платежеспособности организации, вызвать впоследствии значительные расходы по взысканию долгов, а в конечном итоге снизить рентабельность оборотных активов и используемого капитала.

При формировании кредитной политики необходимо определить:
\begin{itemize}
	\item срок предоставления кредита (кредитный период);
	\item размер предоставляемого кредита (кредитный лимит);
	\item стоимость предоставления кредита (система ценовых скидок при осуществлении немедленных расчетов за приобретенную продукцию);
	\item систему штрафных санкций за просрочку исполнения обязательств покупателями.
\end{itemize}

Срок предоставления кредита (кредитный период) характеризует предельный период, на который покупателю предоставляется отсрочка платежа за реализованную продукцию.
Увеличение срока предоставления кредита стимулирует объем реализации продукции (при прочих равных условиях), однако приводит в то же время к увеличению суммы финансовых средств, инвестируемых в дебиторскую задолженность, и увеличению продолжительности финансового и всего операционного цикла организации.
Поэтому, устанавливая размер кредитного периода, необходимо оценивать его влияние на результаты хозяйственной деятельности в комплексе.

Размер предоставляемого кредита (кредитный лимит) характеризует максимальный предел суммы задолженности покупателя по предоставляемому товарному (коммерческому) или потребительскому кредиту.
Он устанавливается с учетом: типа осуществляемой кредитной политики (уровня приемлемого риска); планируемого объема реализации продукции на условиях отсрочки платежей; среднего объема сделок по реализации готовой продукции (при потребительском кредите --- средней стоимости реализуемых в кредит товаров), финансового состояния организации --- кредитора и других факторов.
Кредитный лимит дифференцируется по формам предоставляемого кредита и видам реализуемой продукции.

Стоимость предоставления кредита характеризуется системой ценовых скидок при осуществлении немедленных расчетов за приобретенную продукцию.
Предложение скидок оправдано в следующих ситуациях.
Во-первых, снижение цены приводит к расширению продаж, а затраты, связанные с реализацией продукции, являются несущественными, что отражается на увеличении общей прибыли от продаж.
Во-вторых, если система скидок интенсифицирует приток денежных средств в условиях дефицита в компании.
В-третьих, система скидок за ускорение оплаты более эффективна, чем система штрафных санкций за просроченную оплату.

Система штрафных санкций за просрочку исполнения обязательств покупателями, формируемая в процессе разработки кредитных условий, должна предусматривать соответствующие пени, штрафы и неустойки.
Размеры этих штрафных санкций должны полностью возмещать все финансовые потери организации-кредитора (потерю дохода, инфляционные потери, возмещение риска снижения уровня платежеспособности и др.).

Система контроля движения и своевременного погашения дебиторской задолженности организуется как самостоятельный блок общей системы финансового контроля в организации.
В первую очередь контролируются наиболее крупные и сомнительные виды дебиторской задолженности, затем средние и мелкие, не оказывающие серьезного влияния на общие результаты деятельности организации.
Контроль включает ранжирование дебиторской задолженности по срокам ее возникновения: 0--30 дней, 31--60 дней, 61--90 дней, 91--120 дней, свыше 120 дней.
Особенное внимание уделяется просроченной дебиторской задолженности и причинам ее возникновения

Развитие рыночных отношений и инфраструктуры финансового рынка позволяют использовать в практике финансового менеджмента ряд новых форм управления дебиторской задолженностью --- ее рефинансирование, т. е. ускоренный перевод в другие формы оборотных активов организации: денежные средства и высоколиквидные краткосрочные ценные бумаги.

Основными формами рефинансирования дебиторской задолженности, используемыми в настоящее время, являются: факторинг; учет векселей, выданных покупателями продукции; форфейтинг \cite[361--368]{kirichenko}.

Центральное место в системе управления дебиторской задолженность занимает определение ее допустимого уровня.
Под уровнем понимают ее долю в общей сумме оборотных активов.
Ее определяют по формуле:
 \[ \text{К}_\text{ДЗ} = \dfrac{\text{Средняя за период величина дебиторской задолженности}}{\text{Средняя за период величина оборотных активов}} \] 
 
 Коэффициент $\text{К}_\text{ДЗ}$ имеет вполне определенный смысл, так как показывает долю отвлечения оборотных активов.
 
 Одним из важнейших факторов, которые определяют уровень дебиторской задолженности, является оборачиваемость дебиторской задолженности.
 Она анализируется по следующим показателям:
 \begin{itemize}
 	\item время оборота дебиторской задолженности, по которому определяется вклад дебиторской задолженности в фактическую продолжительность финансового и общего операционного цикла организации;
 	\item скорость оборота средств, инвестированных в дебиторскую задолженность.
 \end{itemize}
 
 Еще одним фактором, определяющим уровень дебиторской задолженности, является качество дебиторской задолженности.
 Его обычно оценивают по следующим показателям:
 \begin{enumerate}
 	\item [а)]коэффициент закрепления дебиторской задолженности ($ \text{КЗ}_\text{{ДЗ}} $) --- показывает, сколько средств отвлекается в дебиторскую задолженность на постоянной основе в расчете на 1 руб. выручки от реализации продукции:
 	\[ \text{КЗ}_\text{{ДЗ}} =  \dfrac{\text{Средняя за период величина дебиторской задолженности}}{\text{Выручка от реализации за период}}\]
 	\item [б)] коэффициент просроченности дебиторской задолженности ($ \text{КП}_\text{{ДЗ}} $) --- показывает долю просроченной задолженности в общей сумме дебиторской задолженности:
	 \[ \text{КП}_\text{{ДЗ}} =  \dfrac{\text{Величина просроченной дебиторской задолженности}}{\text{Средняя за период величина дебиторской задолженности}}\]
	 \item [в)] средний возраст просроченной задолженности ($ \text{ВП}_\text{{ДЗ}} $):
	 \[ \text{ВП}_\text{{ДЗ}} =  \dfrac{\text{Величина просроченной дебиторской задолженности}}{\text{Средний за период однодневный оборот по реализации}}\].
 \end{enumerate}
 
 
 
 
 
 
 
 
 
 
 
 
 
 
 
 