\section{Принципы организации и планирования денежных потоков}

Поддержание оптимальной величины и структуры вложения капитала в денежной форме является ключевой задачей финансового менеджмента.
Целью является получение максимального объема денежных потоков за определенное время.

Процесс управления денежными потоками организации базируется на следующих основных принципах:
\begin{itemize}
	\item принцип информационной достоверности.
	Как и каждая управляющая система, управление денежными потоками организации должно быть обеспечено необходимой информационной базой;
	\item принцип обеспечения сбалансированности.
	Управление денежными потоками организации имеет дело со многими их видами.
	Их подчиненность единым целям и задачам управления требует обеспечения сбалансированности денежных потоков организации по видам объемам, временным интервалам и другим существенным характеристикам.
	Реализация этого принципа связана с оптимизацией денежных потоков организации в процессе управления ими;
	\item принцип обеспечения динамики управления.
	Не все оптимальные управленческие решения, связанные с формированием и использованием денежных средств, разработанные в предшествующем периоде, можно использовать в будущих периодах,	что связано с изменчивостью факторов внешней среды.
	Поэтому при управлении денежными потоками необходимо учитывать потенциал формирования денежных ресурсов, темпы экономического развития, другие внешние факторы;
	\item принцип обеспечения эффективности.
	Денежные потоки организации характеризуются существенной неравномерностью поступления и расходования денежных средств в разрезе	отдельных временных интервалов, что приводит к формированию значительных объемов временно свободных денежных активов организации.
	Это временно свободные остатки денежных	средств, которые теряют свою ценность во времени, от инфляции и по другим причинам.
	Реализация принципа эффективности в процессе управления денежными потоками заключается в обеспечении эффективного их использования путем осуществления финансовых инвестиций организации;
	\item принцип обеспечения разработки нескольких вариантов управленческих решений.
	При наличии альтернативных проектов управленческих решений их выбор должен быть основан на системе критериев, определяющих финансовую стратегию организации;
	\item принцип обеспечения ликвидности.
	Неравномерность отдельных видов денежных потоков порождает временный дефицит денежных средств организации, который отрицательно сказывается на уровне ее платежеспособности.
	Поэтому, в процессе управления денежными потоками необходимо обеспечивать достаточный уровень их ликвидности на протяжении всего рассматриваемого периода.
	Реализация этого принципа обеспечивается путем соответствующей синхронизации положительного и отрицательного денежных потоков в разрезе каждого временного интервала рассматриваемого периода.
\end{itemize}

На основе этих принципов можно определить, что главной целью управления денежными потоками выступает обеспечение финансового равновесия организации в процессе ее развития.
Это достигается путем балансирования объемов поступления и расходования денежных средств за счет их синхронизации во времени \cite[137--140]{kirichenko}.

Планирование денежных потоков включает:
\begin{itemize}
	\item стратегическое планирование, в процессе которого прогноз движения денежных потоков оформляется в форме поступлений и расходов денежных средств по годам планируемого периода и по видам деятельности;
	\item тактическое планирование, в процессе которого разрабатывается план поступления и расходования денежных средств;
	\item оперативное планирование, в процессе которого разрабатывается платежный календарь.
\end{itemize}

В последнее время в российской практике все большее распространение получает \textit{Прогноз движения денежных потоков}.
Этот финансовый документ отражает движение денежных потоков по текущей, инвестиционной и финансовой деятельности.
Такое разграничение по направлениям деятельности позволяет повысить результативность управления денежными потоками.
Прогнозные данные позволяют оценить будущие потоки, а следовательно, перспективы роста организации и ее будущие финансовые потребности \cite[174--175]{kirichenko}.

С помощью прогноза движения денежных потоков можно оценить, сколько денежных средств необходимо вложить в хозяйственную деятельность организации, синхронность поступления и расходование денежных средств, а значит, проверить будущую ликвидность организации.

После составления этого прогноза определяют стратегию финансирования организации, суть которой заключается в следующем:
\begin{itemize}
	\item определении источников долгосрочного финансирования;
	\item формировании структуры и затрат капитала;
	\item выборе способов наращивания долгосрочного капитала.
\end{itemize}

Тактическое финансовое планирование --- это планирование осуществления; оно рассматривается как составная часть стратегического плана и представляет собой конкретизацию его показателей.
Главная задача этого плана --- проверить реальность источников поступления средств (притоков) и обоснованность расходов (оттоков), синхронность их возникновения, определить возможную потребность в заемных средствах.
Это документ, позволяющий реально оценить, сколько денежных средств и в каком периоде потребуется организации.

Среднесрочный план разрабатывается на год с разбивкой по кварталам и подразделениям.
Тактический план по форме соответствует стратегическому плану и служит его развитием и детализацией.